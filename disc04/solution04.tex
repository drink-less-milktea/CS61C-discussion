\documentclass[UTF8,10pt,a4paper]{ctexart}
\usepackage{minted}
\usepackage{amsmath}
\title{Solution of Discussion04}
\author{DrinkLessMilkTea}
\date{\today}
\begin{document}
\maketitle
\section{Pre-Check}
\subsection{}
False, a 寄存器也有可能会改变, 只有 s 寄存器不会
\subsection{}
False, 不一定, 这条指令是将 a0 后 4 个字节为首地址的字加载到 s0 中, 如果 x 数组的元素不是 4 个字节, 则不是 x[1] 被加载到 s0
\subsection{}
True
\subsection{}
True
\subsection{}
False, \textbf{jalr} 是跳转到某个地址并将返回地址存入某个寄存器, 不是直接忽略返回地址, 除非将寄存器设置为 x0 才会忽略返回地址
\subsection{}
False, \textbf{j label} 是指直接跳转到某个 label , 忽略返回地址; 而 \textbf{jal label} 是指跳转到某个 label, 同时将返回地址保存到 ra 寄存器 

\section{RISC-V: A Rundown}
\subsection{}
\begin{minted}{asm}
    addi s0 x0 5    // int x = 5
    sw s0 0(s1)     // y[0] = x0
    mul t0 s0 s0    // int temp = x * x
    sw t0 4(s1)     // y[1] = temp
\end{minted}
\section{Registers}
\subsection{}
\begin{minted}{asm}
    add x8 x0 x11
    or s2 ra t5
\end{minted}
\section{Basic Instructions}
\subsection{}

(a) 
\begin{minted}{c}
    t0 = arr[3] = 4
\end{minted}

(b)
\begin{minted}{c}
    arr[4] = t0 = 4
\end{minted}

(b)
\begin{minted}{c}
    t1 = t0 * 2 = 8
    t2 = s0 + t1 = &arr[2]
    t3 = arr[2] = 3
    t3 = t3 + 1 = 4
    arr[2] = t3 = 4
\end{minted}

(c)
\begin{minted}{c}
    t0 = arr[0] = 1
    t0 = t0 ^ -1 = -2
    t0 = t0 + 1 = -1
\end{minted}
\section{C to RISC-V}
\subsection{}
\subsubsection{}
\begin{minted}{asm}
    li  s0  4
    li  s1  5
    li  s2  6
    li  s3  0
    add s3  s0  s1
    add s3  s3  s2
    addi s3 s3  10
\end{minted}
\subsubsection{}
\begin{minted}{asm}
    sw  x0  0(s0)
    li  s1  2
    sw  s1  4(s0)
    slli s2 s1 2
    add s0  s0  s2
    sw  s1  0(s0)
\end{minted}
\subsubsection{}
\begin{minted}{asm}
    li  s0  5
    li  s1  10
    add t0  s0  s0
    bne t0  s1  ELSE
    li  s1  0
    j   NEXT
ELSE:
    sub s1  s0  1
NEXT:

\end{minted}
\subsubsection{}
\begin{minted}{c}
    for(int i = 0, j = 1;i < 30;i ++) {
        j <<= 1;
    }
\end{minted}
\subsubsection{}
\begin{minted}{asm}
    li  s1  0
loop:
    beq  s0  x0  exit
    add  s1  s1  s0
    addi s0  s0  -1
    j   loop
exit:

\end{minted}
\section{RISC-V with Arrays and Lists}
\subsection{}
\begin{minted}{c}
    arr[2] = arr[0] + arr[1];
\end{minted}
\subsection{}
遍历链表, 将每个元素的值自增 1
\subsection{}
遍历数组, 对每个元素取相反数
\section{RISC-V Calling Conventions}
\subsection{}
通过设定 a 寄存器的值来传递参数, a0, a1, ... 分别代表第 1 个, 第 2 个 ... 参数
\subsection{}
函数的返回值保存在寄存器 a0
\subsection{}
sp 寄存器是栈顶指针, 在函数中, 栈顶指针用来标记当前栈顶的位置, 通过移动栈顶指针可以增加或减少栈空间, 可以用于保存局部变量, 返回地址等
\subsection{}
返回地址, 以及所有后续需要用到的存在 a 和 t 寄存器中的值
\subsection{}
所有后续需要用到的存在 s 寄存器中的值, 以及 sp, gp, tp 寄存器
\subsection{}
s 寄存器和 sp, gp, tp 寄存器保持不变, a 和 t 寄存器还有 ra 寄存器可能改变
\section{Writing RISC-V Functions}
\subsection{}
参数 n 保存在寄存器 a0, \textbf{square} 函数的参数和返回值也保存在 a0, 返回值也存在 a0
\subsection{}
\begin{minted}{asm}
    # Prologue
    addi sp  sp  -16
    sw   ra  0(sp)
    sw   s0  4(sp)
    sw   s1  8(sp)
    sw   s2  12(sp)
    # start
    mv  s0  a0
    bge x0  s0  exit
    j next
exit:
    addi sp sp 
    li  a0  0
    ret
next:
    li  s1  1
    li  s2  0
loop:
    blt s0  s1  end
    mv  a0  s1
    jal square
    add s2  s2  a0
    addi s1 s1  1
    j  loop
end:
    mv  a0  s2
    # Epilogue
    addi sp  sp  16
    lw   ra  0(sp)
    lw   s0  4(sp)
    lw   s1  8(sp)
    lw   s2  12(sp)
    ret
\end{minted}
\section{More Translating between C and RISC-V}
\subsection{}
\begin{minted}{c}
int function(int x, int y) {
    for(int result = 1;y != 0;y --) {
        result *= x;
    }
    return result;
}
\end{minted}
\end{document}